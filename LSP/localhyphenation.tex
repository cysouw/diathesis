%% hyphenation points for line breaks
%% Normally, automatic hyphenation in LaTeX is very good
%% If a word is mis-hyphenated, add it to this file
%%
%% add information to TeX file before \begin{document} with:
%% %% hyphenation points for line breaks
%% Normally, automatic hyphenation in LaTeX is very good
%% If a word is mis-hyphenated, add it to this file
%%
%% add information to TeX file before \begin{document} with:
%% %% hyphenation points for line breaks
%% Normally, automatic hyphenation in LaTeX is very good
%% If a word is mis-hyphenated, add it to this file
%%
%% add information to TeX file before \begin{document} with:
%% %% hyphenation points for line breaks
%% Normally, automatic hyphenation in LaTeX is very good
%% If a word is mis-hyphenated, add it to this file
%%
%% add information to TeX file before \begin{document} with:
%% \include{localhyphenation}
\hyphenation{
    proof-read-ers
    par-a-digm
    equi-pol-lent
    wide-spread
    pre-verb
    role-re-map-pings
    oblig-a-tory
    mero-nym
    mero-nym-ic
    dia-the-sis
    dia-the-ses
    no-mi-na-li-sa-tion
    meta-phor-ical
    de-si-g-na-ting
    spe-ci-fi-cal-ly
    ach-eiv-es
    sta-ges
    warm
    schleu-dern
    ge-ben
    Gar-ten
    ab-stumpf-en
    ge-ring
    ver-mie-ten
    schwa-tz-en
    Frei-tag
    er-hei-tern
    be-schwer-en
    auf-ge-staut-en
    Ber-lin
    Anti-kau-sa-tiv-kon-struk-tio-nen
    Inter-sprach-liche
    Tü-bing-en
    Zu-stands-pas-siv
    Ge-schich-te
    Deut-sche
    Vla-di-mir
    Bil-dungs-be-schränk-ung
    Fun-die-rung
    re-map-ping
    Opi-ni-a-tiv
    kal-ku-lie-ren
    kopf-rech-nen
    auf-zie-hen
    Be-ga-bung
    spa-ren
    kra-chen
    ba-den
    zer-brech-en
    kom-mu-ni-zie-ren
}

\hyphenation{
    proof-read-ers
    par-a-digm
    equi-pol-lent
    wide-spread
    pre-verb
    role-re-map-pings
    oblig-a-tory
    mero-nym
    mero-nym-ic
    dia-the-sis
    dia-the-ses
    no-mi-na-li-sa-tion
    meta-phor-ical
    de-si-g-na-ting
    spe-ci-fi-cal-ly
    ach-eiv-es
    sta-ges
    warm
    schleu-dern
    ge-ben
    Gar-ten
    ab-stumpf-en
    ge-ring
    ver-mie-ten
    schwa-tz-en
    Frei-tag
    er-hei-tern
    be-schwer-en
    auf-ge-staut-en
    Ber-lin
    Anti-kau-sa-tiv-kon-struk-tio-nen
    Inter-sprach-liche
    Tü-bing-en
    Zu-stands-pas-siv
    Ge-schich-te
    Deut-sche
    Vla-di-mir
    Bil-dungs-be-schränk-ung
    Fun-die-rung
    re-map-ping
    Opi-ni-a-tiv
    kal-ku-lie-ren
    kopf-rech-nen
    auf-zie-hen
    Be-ga-bung
    spa-ren
    kra-chen
    ba-den
    zer-brech-en
    kom-mu-ni-zie-ren
}

\hyphenation{
    proof-read-ers
    par-a-digm
    equi-pol-lent
    wide-spread
    pre-verb
    role-re-map-pings
    oblig-a-tory
    mero-nym
    mero-nym-ic
    dia-the-sis
    dia-the-ses
    no-mi-na-li-sa-tion
    meta-phor-ical
    de-si-g-na-ting
    spe-ci-fi-cal-ly
    ach-eiv-es
    sta-ges
    warm
    schleu-dern
    ge-ben
    Gar-ten
    ab-stumpf-en
    ge-ring
    ver-mie-ten
    schwa-tz-en
    Frei-tag
    er-hei-tern
    be-schwer-en
    auf-ge-staut-en
    Ber-lin
    Anti-kau-sa-tiv-kon-struk-tio-nen
    Inter-sprach-liche
    Tü-bing-en
    Zu-stands-pas-siv
    Ge-schich-te
    Deut-sche
    Vla-di-mir
    Bil-dungs-be-schränk-ung
    Fun-die-rung
    re-map-ping
    Opi-ni-a-tiv
    kal-ku-lie-ren
    kopf-rech-nen
    auf-zie-hen
    Be-ga-bung
    spa-ren
    kra-chen
    ba-den
    zer-brech-en
    kom-mu-ni-zie-ren
}

\hyphenation{
    proof-read-ers
    par-a-digm
    equi-pol-lent
    wide-spread
    pre-verb
    role-re-map-pings
    oblig-a-tory
    mero-nym
    mero-nym-ic
    dia-the-sis
    dia-the-ses
    no-mi-na-li-sa-tion
    meta-phor-ical
    de-si-g-na-ting
    spe-ci-fi-cal-ly
    ach-eiv-es
    sta-ges
    speak-er
    move-ment
    need-ed
    cross-ling-uist-ic-al-ly
    cau-sa-tive
    warm
    schleu-dern
    ge-ben
    Gar-ten
    ab-stumpf-en
    ge-ring
    ver-mie-ten
    schwa-tz-en
    Frei-tag
    er-hei-tern
    be-schwer-en
    auf-ge-staut-en
    Ber-lin
    Anti-kau-sa-tiv-kon-struk-tio-nen
    Inter-sprach-liche
    Tü-bing-en
    Zu-stands-pas-siv
    Ge-schich-te
    Deut-sche
    Vla-di-mir
    Bil-dungs-be-schränk-ung
    Fun-die-rung
    re-map-ping
    mne-mo-nic
    Opi-ni-a-tiv
    kal-ku-lie-ren
    kopf-rech-nen
    auf-zieh-en
    Be-ga-bung
    spa-ren
    kra-chen
    ba-den
    kom-mu-ni-zie-ren
    vor-sa-gen
    zer-bre-chen
    über-ge-ben
    schrei-ben
    vor-schla-gen
    quetsch-en
    bre-chen
    auf-stel-len
    bü-geln
    stei-gen
    be-ei-len
    durch-la-vie-ren
    ver-kal-ku-lie-ren
    ver-spe-ku-lie-ren
    ren-tie-ren
    an-eig-nen
    an-zu-schmei-cheln
    du-schen
    vor-stel-len
    be-schäft-ig-en
    kreu-zen
    schlie-ßen
    er-wär-men
    ver-kom-pli-zie-ren
    wie-der-ho-len
    ver-schla-fen
    schick-en
    ge-är-gert
    schmei-ßen
    schwit-zen
    schwin-deln
    ein-fres-sen
    ver-krie-chen
    win-ken
    tank-en
    klopf-en
    ei-sen-be-schla-ge-nen
    schön-ge-schwin-delt
    Par-ti-zip
    ge-schla-fen
    Zeit-lu-pe
    her-ab-ge-stie-gen
    rein-kommt
    ge-blie-ben
    ver-steck-en
    be-kom-men
    mo-da-len
    Po-li-zei-an-ga-ben
    Wahl-kampf
    ge-schlos-sen
    ent-setz-en
    schätz-en
    zu-nehm-en
}
